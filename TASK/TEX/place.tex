%==========================================
%  			EXERCÍCIOS
%===========================================

\begin{flushleft}
	 \textcolor{myBlue}{\textbf{\Large{3 EXERCÍCIOS }}}
\end{flushleft}

\begin{flushleft}
	 \textcolor{myBlue}{\textbf{\large{3.1 Sistema de ordem superior }}}
\end{flushleft}
Para a função de transferência abaixo, determine o tempo de acomodação (\(T_s\)), instante de pico (\(T_p\)) e sobreajuste (\(\% SP\)), assumindo dominância de segunda ordem. Essa suposição é valida?

$$ G(s)=\dfrac{1}{s^3+24,2s^2+109s+500} $$

\noindent \textbf{Resolução: }

Primeiro vemos que temos a equação característica de

\[ s^3 + 24,2s^2  + 109s + 500\] 

Podemos aproximar de uma equação caracerística de segundo grau,

\[ (s^2 + 2\zeta \omega_n s + \omega_n^2)(s+p)  \]

Expandindo essa aproximação e colocando os valores de tempo \(s\) em evidência, temos

\[ s^3 + 2 \omega_n \zeta s^2 + \omega_n^2 s +ps^2 +2 \omega_n \zeta p s + p\omega_n^2  \]
\[ s^3 + (2\omega_n \zeta + p)s^2 + (\omega_n^2 + 2\omega_n \zeta p)s + p\omega_n^2 \]

Com isso temos que

\[ s^3 + 24,2s^2  + 109s + 500 = s^3 + (2\omega_n \zeta + p)s^2 + (\omega_n^2 + 2\omega_n \zeta p)s + p\omega_n^2 \]

Igualando conforme seu termo \(s \) teremos as seguintes expressões

\begin{equation}
24,2 = 2\omega_n \zeta + p \Rightarrow \omega_n \zeta = \dfrac{24,2 - p}{2}
\end{equation}

\begin{equation}
109 = \omega_n^2 + 2\omega_n \zeta p
\end{equation}

\begin{equation}
 500 = p\omega_n^2 \Rightarrow \omega_n^2 = \dfrac{500}{p}
\end{equation}

Trabalhando essas equações podemos achar o valor de \( \omega_n\) e \( \zeta \), começando substituindo a eq. 1 e 3 na eq. 2, temos

\[ 109 = \dfrac{500}{p} + 2 \dfrac{24,2-p}{2}p \]

\[109p = 500 + 24,2p^2-p^3 \]

\[ p^3 - 24,2p^2+109p -500 = 0 \]

\[ \boxed{p_1 = 20 \,\,\mathrm{ou}\,\, p_{2,3}=2,1\pm 4,54i} \]

\newpage

Vamos optar por usar o valor real para achar os valores, ou seja, \(p_1\), com isso temos

\[ \omega_n = \sqrt{\dfrac{500}{p_1}} = \sqrt{\dfrac{500}{20}} \]

\[\boxed{ \omega_n = 5 \,\, \mathrm{rad/s}} \]

\[ \zeta = \dfrac{24,2-p}{2\omega_n} = \dfrac{24,2 - 20}{2\cdot 5} \]

\[ \boxed{\zeta = 0,42} \]

Com esses valores basta aplicar nas fórmulas de sistemas de segunda ordem para acharmos os valores aproximados, começando com \( T_s\) e considerando uma tolerância de \( 2\%\) 

\[ T_s = \dfrac{4}{\omega_n \zeta} = \dfrac{4}{5\cdot 0,42}  \]

\[ \boxed{T_s \approx 1,90 \,\mathrm{s}} \]

Já para \( \% SP \)

\[ \% SP = e^{\frac{-\zeta \pi}{\sqrt{1-\zeta^2}}}\cdot 100 = e^{\frac{-0,42 \pi}{\sqrt{1-0,42^2}}}\cdot 100\]

\[ \boxed{\% SP \approx 23,4\%} \]

E por último \( T_p \)

\[ T_p = \dfrac{\pi}{\omega_n \sqrt{1 - \zeta^2}} = \dfrac{\pi}{5\sqrt{1-0,42^2}} \] 

\[ \boxed{T_p \approx 0,692 \, \mathrm{s}} \]

%
% COMPROVAÇÃO
%
\noindent \textbf{Comprovação: }


Para comprovar se essa resposta está certa, podemos usar ferramentas de software, nesse caso será usado python, primeiro plotando os polos e zeros, o código usado segue:

\begin{lstlisting}
import numpy as np
from scipy import signal
import matplotlib.pyplot as plt

# Definir os numeradores e denominadores
num = [1]  # Numeradores
den = [1, 24.2, 109, 500]  # denominadores

tf = signal.TransferFunction(num, den)

# Calcular Polos e zeros
poles = tf.poles
zeros = tf.zeros

# Plot polos e zeros
plt.figure(figsize=(6, 6))
plt.scatter(np.real(poles), np.imag(poles), marker='x', color='red', label='Poles')
plt.scatter(np.real(zeros), np.imag(zeros), marker='o', color='blue', label='Zeros')
plt.axhline(0, color='black', linewidth=1.7)
plt.axvline(0, color='black', linewidth=1.7)
plt.xlabel('Real')
plt.ylabel('Imaginario')
plt.title('Polos e Zeros')
plt.grid(True)
plt.show()
\end{lstlisting}

Com tal código obtivemos o seguinte plot:

\begin{figure}[H]
  \centering
  \includegraphics[width=0.95\textwidth]{Figure_polos.png}
\end{figure}

Percebe-se  que os valores são exatamente os que calculamos anteriormente, agora comprovado via software. \\

Além disso, podemos plotar as duas funções a inicial (real) e a que aproximamos para ver a diferença, para isso foi usado o seguinte código em python: \vspace{30pt}


\begin{lstlisting}
from scipy import signal
import matplotlib.pyplot as plt

# Definir funcoes de modo num/den

#Funcao de Transferencia Real
num1 = [1]
den1 = [1, 24.2, 109, 500] # s^3 + 24.2s^2 + 109s + 500

# Funcao aproximada de segundo grau
num2 = [1]
den2 = [20, 84, 500] # 20*(s^2 + 4.2s + 25)

# Calculo das funcoes e resposta degrau no dominio do tempo t(s)
lti1 = signal.lti(num1,den1)
time, y_amp1=signal.step(lti1)

lti2 = signal.lti(num2,den2)
time2, y_amp2=signal.step(lti2)

# FT = Funcao de Transferencia
# plot das funcoes
plt.xlabel('Tempo (s)')
plt.ylabel('Resposta ao Degrau')
plt.plot(time, y_amp1, label='FT Real')
plt.plot(time2, y_amp2,label='FT Aproximada')
plt.title('FT Real x Aproximada')
plt.legend()
plt.show()
\end{lstlisting}
\vspace{30pt}

Onde obtivemos o seguinte gráfico, o qual comprova a assertividade do método usado:
\begin{figure}[H]
  \centering
  \includegraphics[width=1\textwidth]{Figure_fts.png}
\end{figure}



\begin{flushleft}
	 \textcolor{myBlue}{\textbf{\large{3.2 Critério de Routh }}}
\end{flushleft}
Considere o sistema em malha fechada mostrado na Figura.
Determine a faixa de valores de K para estabilidade. Assuma que \(K > 0\).

\begin{figure}[H]
\centering
\begin{tikzpicture}

\sbEntree{E}
\sbComp{comp}{E}
\sbRelier[$R(s)$]{E}{comp}
\sbBloc{B}{$K$}{comp}
\sbRelier{comp}{B}
\sbBloc{C}{$\dfrac{(s-2)}{(s+1)(s^2 +6s+25)}$}{B}
\sbRelier{B}{C}
\sbSortie[4]{S}{C}
\sbRelier{C}{S}
\sbNomLien[0.8]{S}{$C(s)$} \sbRenvoi{C-S}{comp}{}

\end{tikzpicture}
\end{figure}

\noindent \textbf{Resolução: }

Deve-se começar simplificando o digrama de blocos para uma só expressão, primeiro faz a multiplicação dos blocos


\begin{figure}[H]
\centering
\begin{tikzpicture}

\sbEntree{E}
\sbComp{comp}{E}
\sbRelier[$R(s)$]{E}{comp}
\sbBloc{B}{$\dfrac{K\cdot (s-2)}{(s+1)(s^2 +6s+25)}$}{comp}
\sbRelier{comp}{B}
\sbSortie[4]{S}{B}
\sbRelier{B}{S}
\sbNomLien[0.8]{S}{$C(s)$}
\sbRenvoi{B-S}{comp}{}

\end{tikzpicture}
\end{figure}

E assim com a realimentação negativa ficamos com
\newpage

\[ \dfrac{C(s)}{R(s)} = \dfrac{\frac{K\cdot (s-2)}{(s+1)(s^2+6s+25)}}{1+ \frac{K\cdot (s-2)}{(s+1)(s^2+6s+25)}\cdot 1} = \dfrac{K(s -2)}{(s+1)(s^2+6s+25)+K(s-2)} \] \\

Simplificando nosso termo \( \frac{C(s)}{R(s)} \), temos \\

\[ \dfrac{C(s)}{R(s)} = \dfrac{K(s-2)}{s^3 + 7s^2 + s(31 + K)+25 -2K} \] \\

ou \\

\begin{figure}[H]
\centering
\begin{tikzpicture}

\sbEntree{E}
\sbNomLien[0.8]{E}{$R(s)$}
\sbBloc[3]{B}{$\dfrac{K(s-2)}{s^3 + 7s^2 + s(31 + K)+25 -2K}$}{E}
\sbRelier{E}{B}
\sbSortie[3]{S}{B}
\sbRelier{B}{S}
\sbNomLien[0.8]{S}{$C(s)$}


\end{tikzpicture}
\end{figure}

Com isso temos que \( K > 0 \) para ser um sistema estável, então a equação

\[ s^3 + 7s^2 + s(31 + K)+25 -2K \]

Fica

\begin{table}[H]
	\centering
    \begin{tabular}{ccc}
        $s^3$ & $1$   & $31+k$ \\
        $s^2$ & $7$   & $25-2k$ \\
        $s^1$ & $b_1$ & $0$ \\
        $s^0$ & $c_1$ &  \\
    \end{tabular}
\end{table}

Após isso basta calcular nossos termos para determinar a faixa

\[ b_1 =  \dfrac{
-\begin{bmatrix}
1 & (31+K) \\
7 & (25-2K)  
\end{bmatrix}
}{7} = \dfrac{-(25-2K-217-7K)}{7} = \dfrac{192+9K}{7} \]

Assim ficamos com a inequação

\[ \dfrac{192+9K}{7} > 0 \]
\[ 192+9K > 0 \]
\[ 9K > -192 \]
\[  K > \dfrac{-192}{9} \]
\[  K > -21,33 \]

Como \( K \), nesse caso ficou negativo, ou seja é instável, deve ser \( K > 0 \).
\newpage

Agora calculamos o outro termo


\[ c_1 =  \dfrac{
-\begin{bmatrix}
7   & (25-2K) \\
b_1 & 0  
\end{bmatrix}
}{b_1} = \dfrac{b_1 (25-2K)}{b_1} = 25-2K \]

Resolvendo a inequação


\[ 25-2K > 0 \]
\[ -2K > -25 \]
\[ 2K < 25 \]
\[ K < \dfrac{25}{2} \]
\[ K < 12,5 \]

Assim fica a resposta final de

\[ \boxed{12,5 > K > 0} \]



\newpage

\begin{flushleft}
	 \textcolor{myBlue}{\textbf{\Large{4 CONCLUSÃO }}}
\end{flushleft}

Sistemas de ordem superior na teoria de controle fornecem uma representação detalhada e realista de sistemas dinâmicos complexos. Compreender e analisar sistemas de ordem superior por meio de analogias, representação de espaço de estados e análise de função de transferência permite que os engenheiros projetem estratégias de controle robustas. \\

O Critério de Estabilidade de Routh é uma ferramenta valiosa na teoria de controle para analisar a estabilidade de sistemas lineares. Ao construir a Tabela de Routh e examinar seus elementos, é possível determinar a estabilidade do sistema sem a necessidade de calcular diretamente os polos. Isso torna o critério uma abordagem eficiente e conveniente para avaliar a estabilidade de sistemas de controle, projetar controladores apropriados e analisar a robustez do sistema em relação a incertezas e variações nos parâmetros.



\newpage

\begin{flushleft}
	 \textcolor{myBlue}{\textbf{\Large{5 REFERÊNCIAS }}}
\end{flushleft}

\noindent OGATA, Katsuhiko. Modern Control Engineering. 5th ed. Upper Saddle River, NJ: Prentice Hall, 2010.\\

\noindent DORF, Richard C.; BISHOP, Robert H. Modern Control Systems. 12th ed. Upper Saddle River, NJ: Prentice Hall, 2010. \\

\noindent NISE, Norman S. Control Systems Engineering. 7th ed. Hoboken, Wiley, 2014.\\

\noindent KUO, Benjamin C. Automatic Control Systems. 10th ed. Hoboken, John Wiley \& Sons, 2017.\\

\noindent PARASKEVOPOULOS, P.N. Modern Control Engineering. 1ª ed. CRC Press, 2001.\\

\noindent The Ryder Project. Disponível em: https://www.youtube.com/@theryderproject5053. Acesso em: 18 jun. 2023.\\

\noindent Sergio A. Castaño. Disponível em: https://www.youtube.com/@SergioACGiraldoBR. Acesso em: 18 jun. 2023.\\

\noindent Controltheoryorg. Disponível em: https://www.youtube.com/@controltheoryorg. Acesso em: 18 jun. 2023.