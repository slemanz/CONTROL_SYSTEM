\noindent \textcolor{myBlue}{\textbf{\large{2 OBJETIVO }}}\\

O projeto do compensador tem como objetivo concluir os seguintes pontos propostos:

\begin{itemize}
\item Definir o número do tipo da função de transferência da planta.
\item Calcular o ganho adicional $K_c$ necessário para atender o requisito de precisão ($E_{ss}$ – erro estático).
\item Calcular a função de transferência G(s) do sistema não compensado, e com o ganho ajustado.
Plotar os gráficos logarítmicos de G($j\omega$) e obter as margens de fase e de ganho.
\item Calcular a defasagem máxima do compensador por avanço de fase, os parâmetros $\alpha$ e T do compensador.
\item Escrever a função de transferência do compensador C(s).
\item Compensar o sistema, desenhar os gráficos logarítmicos do sistema compensado e verificar as margens de fase e de ganho obtidas.
\item Fechar a malha de controle (realimentação unitária e negativa) e obter o gráfico da resposta transitória.
\item Obter o erro estático e compará-lo com o erro estático especificado. 
\end{itemize} 
